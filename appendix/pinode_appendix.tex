\paragraph{Hardware} All experiments were computed using an \texttt{Intel(R) Xeon(R) CPU E5-2630 v4 @ 2.20 GHz} equipped with a \texttt{Tesla K80} GPU. The computer had \texttt{Linux 4.15} installed as an OS.

\subsection{Lifted Duffing Oscillator: Learning Unseen Basins with Collocations (Figure~\ref{fig:duffing_pinode})}
\label{appendix:duffing_pinode}
In this experiment we trained two models; both share the same architecture but differ in the input data, as described in Chapter~\ref{sec:duffing}.

\paragraph{Architecture}
The network consists of two blocks: the autoencoder pair $\phi_\theta(\bd{x}),\ \psi_\theta(\bd{z})$, and the latent dynamics $h_\theta(\bd{z})$, where $\theta$ represents the combined weights of all networks.

Both $\phi$ and $\psi$ were fully-connected networks with three layers, the input-output dimension of 128, the bottleneck-space dimension of 2, and the hidden-layer dimensions of 256. The hidden layers had ReLU activations except for the output layers of $\phi$ and $\psi$ which had linear activation. 

The network $h$ was a fully-connected network with three layers, with the input and output dimension of 2 and the hidden-layer dimensions of 128. All hidden layers had ReLU activation, the output layer had a linear activation.

We used standard network classes of \texttt{pytorch}\footnote{\href{https://pytorch.org}{https://pytorch.org}} by~\cite{pytorch} to implement the networks, and we used a differentiable integrator \texttt{torchdiffeq}\footnote{\href{https://github.com/rtqichen/torchdiffeq}{https://github.com/rtqichen/torchdiffeq}} by~\cite{chen2018neuralode} for evaluating derivatives of the loss function. 

\paragraph{Data} For data \textbf{snapshots} we used 6144 trajectories 10 steps long each, with a step-size $dt = 0.1$. For \textbf{collocations} we generated a set of $10^5$ 2-dimensional points $\bar{\bd{z}}_j \in U\left([-3/2,\, 3/2] \times [-1, 1]\right)$, excluded those belonging to the left (red) lobe, and projected them to the observable space $\mathcal{X}$ using the true decoder~(\ref{eq:duffing_true_decoder}). We then used those high-dimensional points $\bar{\bd{x}}_j$ as collocations. 

\paragraph{Training} We trained the combined model for 400 epochs, with the learning rate of $10^{-4}$. All weights $\omega_i$ were set to 1 for a hybrid model, and $\omega_2$, $\omega_4$ were set to 0 for a data-driven model. 

In each batch we had 64 trajectories and, in case of hybrid models, 640 collocation points. The rationale behind this ratio is that one trajectory contains number-of-steps snapshots of the system. Hence, to balance the amount of information that comes from both sources within a batch, we were taking the number-of-steps more collocations than trajectories for every batch. The same rationale holds for all other instances of training of a hybrid model in this section.

\subsection{Lifted Duffing Oscillator: Far-Out Forecasting (Figure~\ref{fig:duffing_periods}).}
\label{appendix:duffing_forecast}
In this experiment we trained three models, all sharing the same architecture but differ in their loss functions, namely in the coefficients $w_i$.

\paragraph{Architecture} The networks architectures match to the one described in Appendix~\ref{appendix:duffing_pinode}.
\paragraph{Data} For data \textbf{snapshots} we used 6144 trajectories (2048 for each of the attractors) 10 steps long each, with a step-size $dt = 0.1$. For \textbf{collocations} we generated a set of $10^5$ 2-dimensional points $\bar{\bd{z}}_j \in U\left([-3/2,\, 3/2] \times [-1, 1]\right)$ and projected them to the observable space $\mathcal{X}$ using the true decoder~(\ref{eq:duffing_true_decoder}). We then used those high-dimensional points $\bar{\bd{x}}_j$ as collocations. 

\paragraph{Training} We trained the combined model for 400 epochs, with the learning rate of $10^{-4}$. All weights $\omega_i$ were set to 1 for a hybrid model, $\omega_3$, $\omega_4$ were set to 0 for a data-driven model, and $\omega_1 = \omega_2 = 0$ for Physics-Informed model. 

\subsection{Lifted Duffing Oscillator: Role of Non-Linear Latent Dynamics (Figure~\ref{fig:duffing_comparison}).}
\label{appendix:duffing_non_linearity}
In this experiment we trained three models: DMD, PIKN, and PINODE. 

\paragraph{Architecture} The PINODE model's architecture matches to the one described in Appendix~\ref{appendix:duffing_pinode}. The PIKN model's architecture is the same except that the latent dynamics $h(z) = Lz$ is linear: it consists of one fully-connected linear layer of width 16 with no bias and no activation function. DMD model had the latent space of 16; the implementation is faithful to the original works \cite{kutz2016dynamic}.
\paragraph{Data} The datasets for both trajectories and collocations match to the ones described in Apendix~\ref{appendix:duffing_forecast}

\paragraph{Training} We trained the combined model for 400 epochs, with the learning rate of $10^{-4}$. All weights $\omega_i$ were set to 1 for both PIKN and PINODE. DMD model used no collocations, whereas PIKN and PINODE used both trajectories and collocations. 

\subsection{Burgers' Equation: Compressibility (Figure~\ref{fig:burgers_compressibility})}
\label{appendix:burgers_compressibility}
In this section we study how efficiently different ROMs use the same size of the latent space.

\paragraph{Architecture} In \textbf{PINODE} and \textbf{PIKN} models, both $\phi$ and $\psi$ were fully-connected networks with three layers, the input-output dimension of 128 and the hidden-layer dimensions of 512. The hidden layers had ReLU activation except for the output layers of $\phi$ and $\psi$ which had linear activation.
The size of the latent space was varying from 2 to 512, see the x-axis of Figure~(\ref{fig:burgers_compressibility}). 
In \textbf{PINODE}, the network $h$ was a fully-connected network with three layers with the hidden-layer dimensions of 512. All hidden layers had ReLU activation, the output layer had a linear activation. For \textbf{PIKN} the network had one layer of the latent space size with no bias and linear activation. In other words, $h(z) = Az$.
For \textbf{DMD} we used a classic algorithm from~\cite{kutz2016dynamic}, and set the number of DMD modes to be equal to the size of the latent dimension. 
\paragraph{Data}For \textbf{snapshots}, we generated 16384 trajectories of the system for our train dataset and 300 trajectories for our test dataset. Each trajectory had 40 time-steps with $dt = 0.1$. We used randomly-generated functions from Equation~\ref{eq:burger_initial_condition} as initial conditions for the trajectories. We used only first 20 time-steps of train trajectories for training. We used the first 20 steps of the test trajectories to evaluate \textit{interpolation} performance of the model, and the next 20 time-steps for evaluating \textit{extrapolation} performance. All performance measures on Figure~(\ref{appendix:burgers_compressibility}) are based on the test dataset.
For \textbf{collocations}, we used Equation~(\ref{eq:burger_collocations}) to generate $10^5$ collocation points. 
\paragraph{Training} We trained models for 500 epochs, with the learning rate of $10^-4$. All weights $\omega_i$ were set to 1 for a hybrid model, and $\omega_3$, $\omega_4$ were set to 0 for a data-driven model. Every batch contained 64 trajectories and 1280 (that is, 64*20) collocation points, with the same rationale as in Appendix~\ref{appendix:duffing_pinode}.

\subsection{Burger's Equation: Compressibility of Linear Latent Space for PIKN}
\label{appendix:pikn_burgers}
\paragraph{Architecture} 
In the PIKN architecture for Burger's equation, the encoder is implemented as a a feed-forward network with the size of input layer set to be equal 128 (the number of spatial grid-points). It has two hidden layers with 512 neurons. The size of the output layer of the encoder -- the size of the latent dimension -- varies from 16 to 256 to study the latent space compressibility (see Figure~\ref{fig:pikn_burgers_compression}). The decoder's architecture mirrors the architecture of the encoder with the sizes of the inputs and outputs switched. An Adam optimizer has been applied for $500$ epochs for training. We use an adaptive learning rate: initially set to $10^{-4}$, it keeps decreasing by a factor of $0.5$ if no improvement has been made over the recent $20$ epochs, until it hits the minimal value of $10^{-7}$. \\
Each network was trained in two different ways. Namely, we set different values for the weights $\omega_1, \omega_2, \omega_3, \omega_4$ in the loss function, leading to two different training regimes:
\begin{itemize}
    \item $\omega_1=0, \omega_2=0, \omega_3=1, \omega_4=1$. These settings lead to a purely data-driven learning; it corresponds to green lines on Figure~\ref{fig:pikn_burgers_compression}.
    \item $\omega_1=1, \omega_2=1, \omega_3=1, \omega_4=1$. These settings represent a model that uses both data trajectories and collocations (a hybrid model); it corresponds to blue lines on Figure~\ref{fig:pikn_burgers_compression}.
\end{itemize}

\paragraph{Training} 
We sample $80000$ functions $\{u^{(1)}, u^{(2)}, \dots, u^{(80000)}\}$ to use them as collocations. Each function $u^{(j)}$ is a superposition of the Fourier modes that satisfy the periodic boundary conditions. We call them "harmonic" initial conditions, five examples of which are displayed on the middle-right pane of Figure~\ref{fig:pikn_burgers_compression}. We used $\sin{(kx)}$ and $\cos{(kx)}$ for $k=0, 1, 2, \dots$ evaluated on $x \in [-\pi; \pi]$-interval as basis functions. The value of $k$ is restricted to be no greater than $10$. To evaluate $u_t$ at each collocation point we evaluated the spatial derivatives $\{(u_{x}^{(i)}, u_{xx}^{(i)})\}_{i=1}^{80000}$ using a numerical spectral method. 
% \paragraph{Train Data} 

For training trajectories, we use simulation data obtained from a solver base on spectral method. We run the simulation with $1024$ different initial states. Each initial state of $u$ is a Gaussian (bell-) curve with mean 0 and a randomly generated variance $\sigma \sim U(0.1,1)$, evaluated on $[-\pi; \pi]$-interval. Then we sample snapshots of the state $u$ with a temporal gap $\Delta t=0.1$ for $20$ steps (i.e. $p=20$). An example trajectory is displayed on the top-right pane of Figure~\ref{fig:pikn_burgers_compression}.

% \paragraph{Test Data} 
To evaluate and compare the performance of both models we use 200 trajectories: 100 trajectories with harmonic ICs and 100 trajectories with Gaussian ICs generated exactly in the same way as described above. Then we sample snapshots of the state $u$ with a temporal gap $\Delta t=0.1$ for $20$ steps (i.e. $p=20$).

\paragraph{Results} 
The results are aggregated in Table~\ref{table:pikn_burgers_performance} and visualized on Figure~\ref{table:pikn_burgers_performance}.

\begin{table}[t!]
    \centering
    \resizebox{\columnwidth}{!}{\begin{tabular}{llllll}
\toprule
   Latent space & Model &           Train Time &  Test MSE & Test Harmonic MSE & Test Gaussian MSE \\
\midrule
16  & Data-Driven &  0 hours, 18 minutes &   0.04124 &          0.082426 &          0.000054 \\
    & Hybrid &  0 hours, 37 minutes &  0.005451 &          0.010826 &          0.000077 \\
32  & Data-Driven &  0 hours, 19 minutes &  0.039412 &          0.078771 &          0.000054 \\
    & Hybrid &  1 hours, 24 minutes &  0.003815 &           0.00757 &          0.000061 \\
64  & Data-Driven &  0 hours, 21 minutes &  0.036496 &          0.072964 &          0.000028 \\
    & Hybrid &  1 hours, 40 minutes &  0.003513 &          0.006964 &          0.000062 \\
128 & Data-Driven &  0 hours, 22 minutes &  0.033983 &          0.067933 &          0.000033 \\
    & Hybrid &  4 hours, 28 minutes &  0.003334 &          0.006608 &           0.00006 \\
256 & Data-Driven &  0 hours, 38 minutes &  0.035508 &          0.070988 &          0.000028 \\
    & Hybrid &  6 hours, 11 minutes &  0.003396 &          0.006745 &          0.000048 \\
\bottomrule
\end{tabular}
}
    \caption{Performance measures for Burger's experiment depending on the dimensions of the latent space (Figure~\ref{fig:pikn_burgers_compression})}
    \label{table:pikn_burgers_performance}
\end{table}


\subsection{Burgers' Equation: Data-vs-Collocations (Figure \ref{fig:burger_data_vs_collocations})}
In this section we study the relative impact of data and collocations as training datasets to the performance of the resulting models.
\label{appendix:burgers_data_vs_collocations}
\paragraph{Architecture} In \textbf{PINODE} models both $\phi$ and $\psi$ were fully-connected networks with three layers, the input-output dimension of 128, the latent-space dimension of 16 and the hidden-layer dimensions of 512. The hidden layers had ReLU activation except for the output layers of $\phi$ and $\psi$ which had linear activation. The network $h$ was a fully-connected network with three layers with the input-output dimension of 16, and the hidden-layer dimensions of 512. All hidden layers had ReLU activation, the output layer had a linear activation.

\paragraph{Data} For \textbf{snapshots}, we generated 2048 trajectories of the system for our train dataset. Each trajectory had 20 time-steps with $dt = 0.1$. We used randomly-generated functions from Equation~\ref{eq:burger_initial_condition} as initial conditions for the train trajectories. 
For \textbf{collocations}, we used Equation~(\ref{eq:burger_collocations}) to generate $65536$ collocation points. 
We generated 300 trajectories for our test datasets, 100 per kind of initial conditions from Figure~(\ref{fig:burgers_examples_of_ics}). We used all 40 steps of the test trajectories to evaluate the performance of the models. All performance measures on Figure~(\ref{appendix:burgers_data_vs_collocations}) are based on this test dataset.


\paragraph{Training} We trained models for 500 epochs, with the learning rate of $10^-4$. All weights $\omega_i$ were set to 1 for a hybrid model, and $\omega_3$, $\omega_4$ were set to 0 for a data-driven model. Every batch contained 64 trajectories and 1280 (that is, 64*20) collocation points, with the same rationale as in Appendix~\ref{appendix:duffing_pinode}.

\subsection{Burgers' Equation: Robustness to Noise (Figure~\ref{fig:burger_noise})}
In this section we examine robustness to noise for four models: PINODE (Data-Driven, Physics-Informed, Hybrid). We also add DMD to the comparison as a reference point.
\label{appendix:burgers_noise}
\paragraph{Architecture} All PINODE models share the same architecture as described in Appendix~\ref{appendix:burgers_data_vs_collocations}.

\paragraph{Data} We start with the same data as described in Appendix~\ref{appendix:burgers_data_vs_collocations}. Then we apply 11 different levels of Gaussian noise with the mean 0 and the variances distributed log-uniformly between $[10^{-4}, 10^1]$. We only apply noise to the train snapshots; the test snapshots are noise-free. 

\paragraph{Training} We trained models for 500 epochs, with the learning rate of $10^-4$. All weights $\omega_i$ were set to 1 for a hybrid model, $\omega_1$, $\omega_2$ , and $\omega_3$, $\omega_4$ were set to 0 for a data-driven model. Every batch contained 64 trajectories and 1280 (that is, 64*20) collocation points, with the same rationale as in Appendix~\ref{appendix:duffing_pinode}.

