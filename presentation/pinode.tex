\section{Reduced-Order Models}
\subsection{Intro to Reduced-Order Models (ROMs)}

\begin{frame}{Reduced-Order Models}

Schematics, terminology, notation

\end{frame}

\begin{frame}{Variations of ROMs}
	Linear, nonlinear, citations
\end{frame}


\begin{frame}{Physics-Informed Loss}
Definition of Physics-informed ML.

Derivation of PI loss through chain rule. 

Problem with evaluating RHS of differential equations.

Interpretation with cross-borrowing strength. 
\end{frame}

\begin{frame}{Physics-Informed Koopman Networks}
	Linear latent dynamics
	
	Paper
	
	Not a rom so not suitable for downstream tasks. 
\end{frame}

\begin{frame}{Physics-Informed Neural ODEs (PINODE)}
	Nonlinearity in latent space
	- require a lot of data
	- picky to initialization
	- good interpolation bad extrapolation	
\end{frame}

\begin{frame}{PINODE Results on Burgers}
	Add PI loss => Fixes most of the issues
	(example with Burgers)
	Points learned
	- Inform the model about unseen conditions
	Paper reference
\end{frame}

\begin{frame}{PINODE Active}
	> PIKN: "we use it and it works"
	> PINODE: "this is how you need to use it right"
	> PINODE-Active: can we use it right automatically?
	
	Loop scheme
\end{frame}

\begin{frame}{PINODE Active Results}
	Best result (Burgers)
\end{frame}

\begin{frame}{PINODE-CS}
	Application of the above 
	Loss and setup
\end{frame}

\begin{frame}{PINODE-CS Results}
	Results on Burgers
	Maybe results on a harder problem
\end{frame}
