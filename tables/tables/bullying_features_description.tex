\label{appendix:bullying_covariates}
The author acknowledges his colleague and collaborator Damian Santomauro\footnote{\href{mailto:d.santomauro@uq.edu.au}{d.santomauro@uq.edu.au}, Affiliate Assistant Professor of Health Metrics Sciences, Institute for Health Metrics and Evaluation, University of Washington} for providing the dataset, the description of its covariates, and the expert assessment of their historical importance in different rounds of GBD study below.
\begin{enumerate}
	\item \texttt{cv\_symptoms}
	\begin{itemize}
		\item 0 = study assesses participants for MDD or anxiety disorders via a diagnostic interview to determine whether they have a diagnosis. 
		\item 1 = study uses a symptom scale (e.g., Beck Depression Inventory) and uses an established cut-off on that scale to determine caseness. 
		\item Has not  been significant in the past. 
	\end{itemize}
	\item \texttt{cv\_unadjusted}
	\begin{itemize}
		\item 0 = RR is adjusted for potential confounders (e.g., SES, etc.)	
		\item 1 = RR is not adjusted for potential confounders
		\item Has been significant in the past.
	\end{itemize}
	\item \texttt{cv\_b\_parent\_only} 
	\begin{itemize}
		\item 0 = Child is involved in reporting their own exposure to bullying.
		\item 1 = Only parent is involved in reporting the child’s exposure to bullying
		\item Has recently been significant. 
	\end{itemize}
	\item \texttt{cv\_or}
	\begin{itemize}
		\item 0 = estimate is a RR
		\item 1 = estimate is an odds ratio (OR)
		\item ORs are always larger than RRs; covariate may not be significant. 
	\end{itemize}
	\item \texttt{cv\_multi\_reg}
	\begin{itemize}
		\item 0 = RR is the ratio of the rate of the outcome in persons exposed vs all persons unexposed (including persons exposed to low-threshold bullying victimization)
		\item 1 =  RRs are estimated via a logistic regression where exposure represented by 3 categories: 1) No exposure, 2) Occasional exposure, 3) Frequent exposure. The RR for occasional exposure will exclude participants with frequent exposure, and the RR for frequent exposure will exclude participants with occasional exposure. 
		\item Is expected to be significant.
	\end{itemize}
	\item \texttt{cv\_low\_threshold\_bullying}
	\begin{itemize}
		\item 0 = uses a ‘frequent’ exposure  threshold for classing someone as exposed to bullying.
		\item 1 = uses an ‘occasional’ exposure threshold for classing someone as exposed to bullying.
		\item Has been significant in the past. 
	\end{itemize}
	\item \texttt{cv\_anx}
	\begin{itemize}
		\item 0 = estimate represents risk for MDD
		\item 1 = estimate represents risk for anxiety disorders
	\end{itemize}
	\item \texttt{cv\_selection\_bias}
	\begin{itemize}
		\item 0 = < 15\% attrition at followup
		\item 1 = $\geqslant$ 15\% attrition at followup
		\item Has been significant in the past.
	\end{itemize}
	\item \texttt{Percent\_female}
	\begin{itemize}
		\item Indicates \% of sample in estimate that are female. 
	\end{itemize}
	\item \texttt{cv\_child\_baseline}
	\begin{itemize}
	    \item Indicates whether mid-age of sample is above or below 13.
		\item Has not been significant in the past.
	\end{itemize}
\end{enumerate}
